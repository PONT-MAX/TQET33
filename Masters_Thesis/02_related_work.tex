\section{Related work}
In this section important, relevant and fundamental articles to this master's thesis is presented each with a summary. The articles covers compressed sensing theory applied to compressed imaging, SPC architecture and how to evaluate the images i.e. the fundamental source of information on how to build a state of the art SPC system and how to evaluate its performance. 

\subsection{Compressive sensing}
\begin{itemize}


\item \cite{book:sm, book:srr} Two books which thoroughly presents the topic sparse and redundant representation and sparse modeling. The fundamental principles and constraints that needs to be fulfilled in CS. The books presents different minimization algorithms and how to implement them.   

\item In~\cite{article:CS_donoho1} David L. Donoho proposed the framework of compressed sensing and its application to images.

\end{itemize}
\subsection{Compressive imaging}
\begin{itemize}
\item \cite{article:single_pixel_im_cs} "Single-Pixel Imaging via Compressive Sampling"
\item \cite{article:cs_for_prac_ios_a_tut} "Compressed sensing for practical optical imaging systems: a tutorial"

\item \cite{article:a_new_ci_arc} "A New compressive imaging camera architecture using Optical-Domain Compression"

\item \cite{article:an_Arcitecture_for_CI} "An architechture for compressive imaging"

\item \cite{article:a_high_res_swir} "A high resolution SWIR camera via compressed sensing"

\item \cite{article:cs_for_prac_ios_a_tut} "Compressive Sensing: From Theory to Applications, A survey"

\item \cite{mt:EF} "Compressed Sensing for 3D Laser Radar"  

\item \cite{article:misuperres} "Multi image super resolution using compressed sensing" 
	
\end{itemize}
    
\subsubsection{Measurement matrix \& reconstruction}

\begin{itemize}
    
    \item \cite{article:TVAL3} Chengbo Li:s master's thesis "An Efficient Algorithm For Total Variation Regularization with Applications to the Single Pixel Camera and Compressive Sensing" describes his new total variation algorithm Li constructed which solve the CS problem. The algorithm is faster and produces better results for images than previous popular algorithms.   

    \item \cite{article:SRM_short, article:SRM_long, article:SRM_block} Fast and Efficient Compressive Sensing Using Structurally Random Matrices (SRM). The articles describes why and how to implement SRM, in these articles the Hadamard or DCT matrices is proposed to replace the i.i.d random matrix. With SRM the reconstruction time is reduced by replacing matrix multiplication with fast transforms. In addition to improved reconstruction time the new method does not need to store the measurement matrix in memory which enables reconstruction of high resolution images. 

	\item \cite{article:an_improved_WH_matrix} "An Improved Hadamard Measurement Matrix Based on Walsh Code For Compressive Sensing" Shows that sequency-ordered Walsh Hadamard matrix gives better reconstruction then the Hadamard matrix with the same benefits of using the Hadamard matrix. The resulting reconstructed image has near optimal reconstruction performance.

\end{itemize}


\subsection{Evaluation}

\begin{itemize}

    \item Al Boviks book the essential guide to image processing \cite{book:image_processing} contains the majority of fundamental image processing techniques and measurements. Two image quality metrics of interest is PSNR and SSIM which can be used when a reference image is available.
    
    \item \cite{article:il_niqe} A Feature-Enriched Completely Blind Image
    Quality Evaluator describes how the no reference image quality assessment tool IL-NIQE works and compares to other NR-IQA algorithms. In the article a comparison with other state of the art NR-IQA is conducted which concludes that IL-NIQE is the best over all NR-IQA tool. This kind of QA is useful when there is no reference image available, which is true when taking photos with the SPC.
    
    
    
\end{itemize}

% \begin{itemize} ctrl /
%     \item Image processing - SSIM & LSM
%     \item TU - BLIINDS, BRISQUE, NIQE
% \end{itemize}

\subsection{Analysis}