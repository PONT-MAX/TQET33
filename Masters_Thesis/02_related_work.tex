\section{Related work}
\label{sec:related_work}
In this section important, relevant and fundamental articles to this master's thesis is presented each with a summary. The articles covers compressed sensing theory applied to compressed imaging, SPC architecture and how to evaluate the images i.e. the fundamental source of information on how to build a state of the art SPC system and how to evaluate its performance. 

\subsection{Compressive sensing}
\begin{itemize}


\item \cite{book:sm, book:srr} "Sparse Modeling" by G. Y. Grabarnik and  I. Rish and "Sparse and redundant representation" by M Elad is two books which thoroughly presents the topic of sparse and redundant representation and modeling. The fundamental principles and constraints that needs to be fulfilled in CS are described. The books presents different minimization algorithms and how to implement them.   

\item In~\cite{article:CS_donoho1} by "Compressed sensing" David L. Donoho proposed the framework of compressed sensing and the application of images capturing.

\item \cite{article:CS_from_theory_a_sur} "Compressive Sensing: From Theory to Applications, A survey" by S. Qaisar et al. 2013, reviews CS background, theory and mathematics and has a extensive survey of reconstruction algorithm and potential CS applications.

\end{itemize}
\subsection{Compressive imaging}
\begin{itemize}

\item \cite{article:an_Arcitecture_for_CI,article:a_new_ci_arc} "An architechture for compressive imaging" and "A New compressive imaging camera architecture using Optical-Domain Compression" by M. B. Wakin, D Takhar, et al. 2006, presents the first single pixel camera architecture using CS to reconstruct the images. 

\item \cite{article:single_pixel_im_cs} "Single-Pixel Imaging via Compressive Sampling" by M. F. Duarte et al. 2008, is an introduction and summary to CS and CI including the SPC architecture. This article also compares different scanning methodologies and their conditions.   

\item \cite{article:foiSPIS} "Single Pixel SWIR Imaging using Compressed Sensing" by C. Brännlund and D. Gustafsson, 2016, shows the initial results and proof of concept of the SPC architecture at FOI.
  
%\item \cite{article:cs_for_prac_ios_a_tut} "Compressed sensing for practical optical imaging systems: a tutorial"


\item \cite{article:a_high_res_swir} "A high resolution SWIR camera via compressed sensing" is a paper by L. McMackin et al. 2012 at Inview Technology which develop and produces compressive sensing cameras. The paper contains a brief review of CS and CI followed by a presentation of their camera architecture. 

 
\item \cite{mt:EF} "Compressed Sensing for 3D Laser Radar" by E. Fall, 2014, is a master's thesis where CS/CI is evaluated for a potential depth camera architecture using a one pixel sensor.   

\item \cite{article:misuperres} "Multi image super resolution using compressed sensing" by T. Edler et al. 2011, presents the results from using a small detector array instead of just one single sensor, but still using CS to reconstruct the images. With this technique the subsampling ratio and the exposure time is reduced. 

 
	
\end{itemize}
    
\subsection{Measurement matrix \& reconstruction}

\begin{itemize}
    
\item \cite{article:TVAL3} Chengbo Li:s master's thesis "An Efficient Algorithm For Total Variation Regularization with Applications to the Single Pixel Camera and Compressive Sensing" describes a total variation algorithm that Li constructed which solves the CS problem. The algorithm is faster and produces better results for images than previous popular algorithms.   

\item \cite{article:SRM_short, article:SRM_long, article:SRM_block} Fast and Efficient Compressive Sensing Using Structurally Random Matrices (SRM). The articles describes why and how to implement SRM, in these articles the Hadamard or DCT matrices is proposed to replace the random measurement matrix. With SRM the reconstruction time is reduced by replacing matrix multiplication with fast transforms. In addition to improve reconstruction time, the new method does not need to store the measurement matrix in memory, which enables reconstruction of high resolution images. 

\item \cite{article:an_improved_WH_matrix} "An Improved Hadamard Measurement Matrix Based on Walsh Code For Compressive Sensing" shows that sequency-ordered Walsh Hadamard matrix gives better reconstruction than the Hadamard matrix with the same benefits of using the Hadamard matrix. The resulting reconstructed image has near optimal reconstruction performance.
	

\end{itemize}


\subsection{Evaluation}

\begin{itemize}

\item \cite{book:image_processing} "The essential guide to image processing" by Al Boviks contains the majority of fundamental image processing techniques and measurements. Two image quality metrics of interest is PSNR and SSIM which can be used when a reference image is available.
    
\item \cite{article:brisque} "No-Reference Image Quality Assessment
in the Spatial Domain" by M. Anish et al. 2012, is the article describing the blind/referenceless image spatial quality evaluator (BRISQUE). The BRISQUE algorithm evaluates image quality and “naturalness” based on statistics in the image. BRISQUE is used when there is no reference image available and therefore can be used to evaluate images produced by the SPC.  
    
\item \cite{article:FOI_pres_sens} "Prestandamått för sensorsystem" by F. Näsström et al. 2016, describes methods and tools to evaluate sensor systems at FOI. 
    
\end{itemize}



%\subsection{Analysis}


