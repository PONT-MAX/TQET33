

Photos captured in the shortwave infrared (SWIR) spectrum are interesting in military applications because they are independent of what time of day the picture is captured because the sun, moon, stars and night glow illuminate the earth with short-wave infrared radiation constantly. A major problem with today's SWIR cameras is that they are very expensive to produce and hence not broadly available either within the military or to civilians. Using a relatively new technology called compressive sensing (CS), enables a new type of camera with only a single pixel sensor in the sensor (a SPC). This new type of camera only needs a fraction of measurements relative to the number of pixels to be reconstructed and reduces the cost of a short-wave infrared camera with a factor of 20. The camera uses a micromirror array (DMD) to select which mirrors (pixels) to be measured in the scene, thus creating an underdetermined linear equation system that can be solved using the techniques described in CS to reconstruct the image. Given the new technology, it is in the Swedish Defence Research Agency (FOI) interest to evaluate the potential of a single pixel camera. With a SPC architecture developed by FOI, the goal of this thesis was to develop methods for sampling, reconstructing images and evaluating their quality. This thesis shows that structured random matrices and fast transforms have to be used to enable high resolution images and speed up the process of reconstructing images significantly. The evaluation of the images could be done with standard measurements associated with camera evaluation and showed that the camera can reproduce high resolution images with relative high image quality in daylight.
