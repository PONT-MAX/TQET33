Foton tagna i det korta infraröda spektrumet är intressanta i militära sammanhang på grund av att de är mindre beroende av vilken tid på dygnet de är tagna för att solen, månen, stjärnor och nattsken (night glow) lyser upp jorden med kortvågiga infraröd strålning dynget runt. Ett stort problem med dagens kortvågig infraröda kameror är att de är väldigt dyra att producera och därav inte tillgängliga till en bred skara, varken militärt eller civilt. Med hjälp av en relativt ny teknik kallad \textit{compressive sensing} (CS) möjligörs en ny typ av kamera med endast en pixel i sensorn. Denna nya typ av kamera behöver bara en bråkdel mätningar relativt antal pixlar som ska återskapas och reducerar kostnaden på en kortvågig infraröd kamera med en faktor 20. Kameran använder en mikrospegelmatris som används för att välja vilka speglar (pixlar) som ska mätas i scenen och på så sätt skapa ett underbestämt ekvationssystem som kan lösas med teknikerna beskrivna i CS för att återskapa bilden. Givet den nya tekniken är det i Totalförsvarets forskningsinstituts (FOI) intresse att utvärdera potentialen hos en enpixel-kamera. Med en enpixel-kameraarkitektur utvecklad av FOI var målet med detta examensarbete att ta fram metoder för att sampla, återskapa bilder och utvärdera deras kvalitet. Detta examensarbete visar att användning av strukturella slumpade matriser och snabba transformer öppnar upp för högupplösta bilder och snabbar upp processen att rekonstruera bilder avsevärt. Utvärderingen av bilderna kunde göras med vanliga mått associerade med kamerautvärdering och visade att kameran kan återskapa högupplösta bilder med relativt hög bildkvalitet i dagsljus. \citep{article:FOI_pres_sens}