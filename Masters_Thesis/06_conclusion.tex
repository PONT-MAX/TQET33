\section{Conclusion \& Future Work}
\label{sec:conclusions_and_fw}
Explicit answers to research questions and a summary of the whole masters thesis.\\[0.1in]

In this thesis a complete compressive imaging SWIR SPC architecture was implemented and evaluated. The aim was to find out which image quality could be achieved in natural images captured in daylight, the results produced in thesis both presented evaluation from simulated images and images captured by the SPC to show both how the chosen sampling method and reconstruction algorithm performed and how the whole architecture performed in unison.\\[0.1in]

The sampling strategy using the structural random matrices method with sequency ordered Walsh Hadamard measurement matrix solved the problems of scaling the reconstructed images to high definition photos and enabled the image resolution $512 \times 512$ pixels in this thesis, the same resolution as the state of the art reference SWIR camera.\\[0.1in]

The total variation solver TVAL3 was used as reconstruction algorithm  which took advantage of SRM with a FWHT to reconstruct the images fast and with good preservation of edges. Using the chosen sampling and reconstruction method could potentially make for a very lightweight sampling and reconstruction procedure with few variables stored in memory and calculations made in real time.\\[0.1in]

Feeding the measurement matrices to the DMD through a HDMI cable and video software was an easy method to start with. The first set of measurements was performed with low frame rate and blank frames for control. But when maximizing the frame rate the risk of duplicate or frame drops increased which made it hard to par each measurement to the correct measurement matrix. 

In post processing an algorithm to correct the signal effected by luminance change was implemented, this algorithm showed to of outmost importance to this thesis. Long exposure time in natural scenes always gave a significant DC change in the sampled signal which should be stationary. Result showed in both simulation of luminance change and in real scenes a significant improvement of image quality when using this method and the analysis indicate that the algorithm may be relevant even if the exposure time gets reduced to near a second.\\[0.1in]

The resulting images produced by the SPC shows that high quality and high resolution images can be acquired in natural daylight scenes. In good conditions with enough intensity to overcome the sensors background noise and relatively stationary scenes detailed images where even people could be identified could be reproduced. Given the result and further work and improvement if the hardware compressive imaging and the SPC architecture have potential in real world applications.\\[0.1in]

The resulting reconstructed images from both simulations and the SPC was evaluated with a range of methods. When a reference image was available standard image evaluation techniques, PSNR and SSIM was used. In the subsampling range 5-30\% as was used through out the thesis showed that image quality increased with increasing subsampling ratio but stagnated around subsampling ratio between 15-20\%. The BRISQUE no reference image quality assessment algorithm indexed based on statistics of "naturalness" in the images and it could be seen that the SPC could get the same results as the best in the simulations indicating that the sampling and reconstruction is the main source of image degradiation and that the SPC hardware in the right conditions do not effects the resulting image quality negatively.\\[0.1in] 

The edge response algorithm 




   

\begin{itemize}
    \item How can the quality of images reconstructed by CS or a SPC be evaluated?
    \item What is the state of the art method to capture and reconstruct images using a SPC architecture?
    \item What image quality is achieved using state of the art methods applied to the SPC?
\end{itemize}

What image quality can be achieved in natural images captured with a single pixel camera in daylight using state of the art methods? 

\subsection{Conclusion}





\subsection{Future work}
This thesis shows that there is possible to use the SPC architecture to capture and reconstruct natural scenes in daylight, but for the technology to be used in a realistic application some improvements need to be made. And I think the most crucial problems could be "bought out" by more sophisticated hardware.\\[0.1in]

As identified multiple times in this thesis, the largest contributor to decreased image quality is exposure time and noise. The exposure time can be decreased by a faster DMD, today there exist a DMD:s with 8 times the maximum pattern rate of the DMD used in this thesis (which was operated at half the maximum speed), which would enable an exposure time of 1.125 seconds at 10\% subsampling ratio. This upgrade would however require a new approach to feeding the measurements matrices to the DMD because of the limitations of the HDMI cables transfer speed.\\[0.1in]

Ether if a new transfer approach is implemented or not, a more sophisticated method of generating each measurement matrix should be implemented, the current method of generating all  measurement matrices offline to a video file and then playbacked by a third Party video player has a limitation of FPS but also a reliability problem. The software was not designed to necessarily show every frame in the video file, which is a problem where each measurement needs to be paired with the correct measurement matrix. My suggestion would be a program which would calculate all matrices at start up and upload them to video memory ready for to display at the DMD. This solution would also be more agile where subsampling ratio, which type of measurement matrix and exposure time could easily be adjusted prior to sampling.\\[0.1in]

The second hardware upgrade, to reduce noise, would be to replace the photo diode used in this thesis to a more appropriate sensor for the application. The new sensor should have less background noise and the surface area of the doid should be analyzed to match the rest of the architecture to maximize the incoming light onto the sensor through the focusing lens. This upgrade could also unlock the true potential of the SWIR spectrum where images could be captured in dark environments illuminated by the moon, stars and night glow.\\[0.1in]

New research shows promising results by adding a second sensor which measures the intensity of the mirrors representing zero or turned away from the sensor and being dumped in the current architecture. The compliment of each unique measurement matrix could also be seen as a unique measurement matrix and thus more information is collected in each measurement and thus reduce the subsampling ratio needed to reconstruct a corresponding single sensor image.\cite{article:nature_dual_sensor}\\[0.1in]

The last future work proposition will not increase image quality but would simplify research or usage and time consumption taking pictures with the SPC. The second and third most time consuming task after exposure time is the complexity of sampling the scene and the reconstruction algorithm in the post process. A fully integrated capturing and reconstruction chain with a gpu accelerated reconstruction algorithm should ease the work and save a lot of time for the user. 
 
