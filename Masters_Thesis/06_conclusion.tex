\chapter{Conclusion \& Future Work}
\label{sec:conclusions_and_fw}
%Explicit answers to research questions and a summary of the whole masters thesis.\\[0.1in]
\section{Conclusion}
\label{sec:conclusion}

In this thesis a complete compressive imaging SWIR SPC architecture was implemented and evaluated. The aim was to find out which image quality could be achieved in natural images captured in daylight. The results produced in this thesis both presented evaluation from simulated images and images captured by the SPC to show how the chosen sampling method and the reconstruction algorithm performed and how the whole architecture performed in unison.\\[0.1in]

The sampling strategy using the structural random matrices method with sequency ordered Walsh Hadamard measurement matrix solved the problems of scaling the reconstructed images to high definition photos and enabled the image resolution $512 \times 512$ pixels, which is, the same resolution as the state of the art reference SWIR camera.\\[0.1in]

The total variation solver TVAL3 was used as reconstruction algorithm  which took advantage of SRM with FWHT to reconstruct the images fast and with good preservation of edges. Using the chosen sampling and reconstruction method could potentially make for a lightweight sampling and reconstruction procedure with few variables stored in memory and calculations made in real time.\\[0.1in]

Feeding the measurement matrices to the DMD through a HDMI cable and video software was an easy method to start with. The first set of measurements was performed with low frame rate and blank frames for control. But when maximizing the frame rate the risk of duplicate or frame drops increased which made it hard to pair each measurement to the correct measurement matrix. This method need to be replaced for more control and faster sampling rates.\\[0.1in]

For post-processing, an algorithm to correct the signal affected by luminance change was implemented. This algorithm showed to be of utmost importance to this thesis. Long exposure time in natural scenes always gave a significant DC change in the sampled signal which should be stationary. Results showed in both simulations of luminance change and real scenes a significant improvement of image quality when using this method. The analysis also indicate that the algorithm may be relevant even if the exposure time gets reduced to near a second.\\[0.1in]

The resulting images produced by the SPC showed that high quality and high resolution images can be acquired in natural daylight scenes. In good conditions with sufficient intensity to overcome the sensors background noise and relatively stationary scenes, detailed images where even people could be identified, could be reproduced. Given the result and further work and improvement of the hardware, compressive imaging and the SPC architecture have potential of real world applications.\\[0.1in]

The resulting reconstructed images from both simulations and the SPC was evaluated with a range of methods. When a reference image was available, standard image evaluation techniques, PSNR and SSIM was used. Experiments in the subsampling range 5-30\% that was used throughout the thesis, showed that image quality increased with increasing subsampling ratio, but stagnated around a subsampling ratio between 15-20\%. The BRISQUE no reference image quality assessment algorithm based on statistics of "naturalness" in the images proved usefull. It could be seen that the SPC could get the same results as the best results in the simulations. This indicates that the sampling and reconstruction is the main source of image degradation and that the SPC hardware in right conditions does not affect the resulting image quality negatively.\\[0.1in] 

The edge response algorithm calculates the sharpness of the image and therefore comparison between cameras and image processing methods is easily performed. This evaluation showed that the reference SWIR camera performs a bit better than the SPC. Because the SWIR SPC is a completely different camera than a conventional camera and has a different purpose and application than a regular camera a good evaluation is also to present the produced images to get a subjective view of the quality and what is good enough.\\[0.1in]


To summarize the whole thesis, the research questions are answered:


\begin{itemize}
    \item The image quality of the produced images can be evaluated with the same methods used when evaluating a regular camera. In the case of the SPC, the sampling matrix and reconstruction algorithm can be evaluated independently of the SPC hardware as well.
    
    \item The state of the art method to capture and reconstruct images using an SPC architecture is to sample all the measurement matrices as fast as possible using structurally random matrices and fast transforms in the reconstruction algorithm.
    
    \item The image quality achieved using state of the art methods applied to the SPC is high resolution images with good enough quality to be used in real world applications. 
\end{itemize}







\section{Future work}
This thesis shows that there is possible to use the SPC architecture to capture and reconstruct natural scenes in daylight, but for the technology to be used in a realistic application some improvements should be made. I think that the most crucial problems will disappear by using more sophisticated hardware.\\[0.1in]

As identified multiple times in this thesis, the largest contributors to decreased image quality are exposure time and noise. The exposure time can be decreased by a faster DMD, today there exist DMD:s with 8 times the maximum pattern rate of the DMD used in this thesis (which was operated at half the maximum speed), which would enable an exposure time of 1.125 seconds at 10\% subsampling ratio. This upgrade would however require a new approach to feeding the measurements matrices to the DMD because of the limitations of the HDMI cables transfer speed.\\[0.1in]

Either a new transfer approach is implemented or not, a more sophisticated method of generating each measurement matrix should be implemented. The current method of generating all  measurement matrices offline to a video file and then playbacked by a third party video player, has a limitation of FPS but also a reliability problem. The software was not designed to necessarily show every frame in the video file, which is a problem for the SPC, where each measurement needs to be paired with the correct measurement matrix. My suggestion would be a program which would calculate all matrices at start up and upload them to video memory ready to display at the DMD. This solution would also be more agile where, subsampling ratio, type of measurement matrix and exposure time could easily be adjusted prior to sampling.\\[0.1in]

The second hardware upgrade with the aim to reduce noise, would be to replace the photo diode used in this thesis to a more appropriate sensor for the application. The new sensor should have less background noise and the surface area of the diode should be analyzed to match the rest of the architecture to maximize the incoming light onto the sensor through the focusing lens. This upgrade could also unlock the true potential of the SWIR spectrum where images could be captured in dark environments illuminated by the moon, stars and night glow.\\[0.1in]

New research shows promising results by adding a second sensor which measures the intensity of the mirrors representing zero or turned away from the sensor and being dumped in the current architecture. The compliment of each unique measurement matrix could also be seen as a unique measurement matrix and thus more information can be collected in each measurement and reduce the subsampling ratio needed to reconstruct a corresponding single sensor image.\cite{article:nature_dual_sensor}\\[0.1in]

The last future work proposition will not increase image quality but would simplify research or usage and time consumption taking pictures with the SPC. The second and third most time consuming task after exposure time is the complexity of sampling the scene and the reconstruction algorithm in the post-process. A fully integrated capturing and reconstruction chain with a GPU accelerated reconstruction algorithm should ease the work and save a lot of time for the user. 
 
