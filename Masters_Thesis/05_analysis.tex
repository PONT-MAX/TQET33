\section{Discussion}
In this section an analysis and discussion is held to describe how to interpret and explain the result from the evaluation section given the theory in related work.

\subsection{Result} 
%Finns det något i resultaten som står ut och behöver analyseras och kommenteras? Hur förhåller sig resultaten till det material som togs upp i teorigenomgången? Vad säger teorin om vad resultaten egentligen betyder? Vad innebär det till exempel att man vid en användbarhetsmätning av ett nytt system fått ett visst värde; hur bra eller dåligt är det? Finns det något i resultaten som är oväntat baserat på teorigenomgången, eller stämmer det bra överens med vad man teoretiskt kunde förvänta sig? 


\subsection{Evaluation Using reference image}
%Sec:reconstruction performance using ref im
%A second observation is that, when the noise increases the  reconstructed image quality is not improved at the same rate as the noiseless case when increasing the sub sampling ratio.
meaning that a bad signal can not be salvaged by more measurements.



\subsubsection{Evaluating Using BRISQUE}
%In figure~\ref{fig:Brisque_2d} the result has been flatten to a 2D graph with fewer selected data points for clarity. Even in the noiseless case the score will not be better then approximately 40 for the reconstructed images while the SWIR images has a mean value of 15. 
The conclusion from this result is that, with the current measurement matrix and reconstruction method around 40 in BRISQUE score is what to expect as optimum given that the SPC will induce noise to the signal. 

\subsubsection{Number of measurements}

\subsubsection{Edge response}

\subsubsection{Dynamics in scene}
%As seen in figure~\ref{fig:local_sig_1} there is no obvious difference between the non perturbed reference signal and the distorted signal. In figure~\ref{fig:local_sig_2} where some of the samples is displayed no large difference can be seen either. It can be interpreted as added noise to the signal and it is barely detectable even if the signal is known.

%The conclusion of this test implies that local movement in a scene will cause noise in the image globally and especially locally where the movement occurred. It also implies that local movement is very hard to detect in the signal even if a reference signal is available.\\[0.1in] 

%Obviously in this context the samples with movement is very easy to spot and the easiest fix would be to just remove those measurements, reconstructing an image with fewer measurements. The resulting image would not be as good as the image in figure~\ref{fig:fly_2} but it would not be contain the noise present in figure~\ref{fig:fly_3}.\\[0.1in]


Even when the Exposure time is reduced to a second there is some indication that the Luminance change will still effect the result (k corresponds to 50 ms) so it might be a good idea for faster SPC to use this method also.

\subsection{Method} %Vad jag tycker om det (personliga tankar)

%Här ska den använda metoden diskuteras och kritiseras. Att ha ett kritiskt förhållningssätt till använd metod är en viktig del av vetenskaplighet.

%En studie är sällan perfekt. Det finns nästan alltid saker man skulle vilja gjort annorlunda om man kunnat göra om studien eller haft extra resurser. Gå igenom de viktigaste bristerna du ser med din metod och diskutera tänkbara konsekvenser för resultaten. Koppla tillbaka till den metodteori som togs upp i teorikapitlet. Referera explicit till relevanta källor. 

% Reliabilitet är ett begrepp som beskriver mätningens kvalitet: hur väl kan man lita på data som insamlats och hur det används för att dra slutsatser. Om reliabiliteten är hög, så bör samma/liknande resultat kunna uppnås om man gör om studien med samma metod.

%Validitet handlar lite förenklat om huruvida man i en mätning mätt det man tror sig mäta. En studie med hög validitet har alltså en hög grad av trovärdighet.

\subsection{Work in a Broader context}
%Det ska ingå ett stycke med en diskussion om etiska och samhälleliga aspekter relaterade till arbetet. Detta är viktigt för att påvisa professionell mognad samt för att utbildningsmålen ska kunna uppnås. Om arbetet av någon anledning helt saknar koppling till etiska eller samhälleliga aspekter ska detta explicit anges i stycket Avgränsningar i inledningskapitlet. 

% Jag tror detta är fallet då det är en ny sorts kamera, det finns väl inget morariskt fel att bygga en kamera som är effektivare? och den borde inte få några samhelliga effekter?
